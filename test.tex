\documentclass[a4paper]{article}
\usepackage[utf8]{inputenc}
\usepackage[T2A]{fontenc}
\usepackage[english,russian]{babel} 
\usepackage[left=25mm, top=20mm, right=25mm, bottom=30mm, nohead, nofoot]{geometry}
\usepackage{amsmath,amsfonts,amssymb} % математический пакет
\usepackage{fancybox,fancyhdr} 
\pagestyle{fancy}
\fancyhf{}
\fancyhead[R]{\href{https://vk.com/wildmathing}{Wild Mathing}}
\fancyfoot[R]{\thepage} 
\fancyhead[L]{Asymptote}
\setcounter{page}{2} % счетчик нумерации страниц
\headsep=10mm 
\usepackage{xcolor}
\usepackage{hyperref} 
\hypersetup{colorlinks=true, allcolors=[RGB]{010 090 200}} % цвет ссылок 
\usepackage{graphicx} % для картинок
\usepackage{asymptote} % создание рисунков
\usepackage{comment} % комментирование
\setlength{\footskip}{12.0pt}
\setlength{\headheight}{12.0pt}

\begin{document}

\section*{Примеры иллюстраций и кода Asymptote}
\subsection*{Полезные ссылки}
\href{https://vk.com/wall-201568161_2898}{Видеоурок} \\
\href{https://vk.com/wall-201568161_2821}{Онлайн-редактор} \\ 
\href{https://vk.com/wall-201568161_2998}{Стереометрия} 

\subsubsection*{Вписанная и описанная окружность}
\begin{asy}
size(7cm);
import geometry;
triangle t=triangleabc(3, 4, 6);
draw(t); label(t); draw(incircle(t));
clipdraw(circumcircle(t));
\end{asy}


\subsubsection*{Вектор на клетчатом поле}
\begin{asy}
size(5cm);
import geometry;
add(grid(8,5, grey));
pair A=(1,1), B=(7,4);
draw(A--B, blue+1bp, EndArrow(HookHead, 2mm));
label("$A$", A, SW);
label("$B$", B, NE);
\end{asy}


\subsubsection*{Маршрут с двумя переправами}
\begin{asy}
size(6cm,0);
import geometry;
defaultpen(fontsize(10pt));
dotfactor=8;
pair A, B, K, L, M, P;
A=(1,1); B=(9,9); K=(2,3); L=K+N; M=(5,5); P=M+N;
add(grid(10,10, mediumgray));
path square=shift(3N)*xscale(10)*unitsquare;
axialshade(square, lightblue, 10E, paleblue, W);
axialshade(shift(2N)*square, lightblue, 10E, paleblue, W);
draw(A--K--L--M--P--B, black+1bp);
distance(scale(0.9)*"$10$", (0,0), 10E, Arrows(HookHead, 1.25mm));
dot("$A$", A, SW, UnFill); 
dot("$B$", B, NE, UnFill);
\end{asy}

\subsubsection*{Самый короткий код}
\begin{asy}
size(4cm); draw(polygon(6)^^unitcircle);
\end{asy}

\subsubsection*{Тригонометрическая спираль}
\begin{asy}
import graph;  // для отрисовки осей и графиков 
size(6cm); // размер рисунка
texpreamble("\usepackage{amsmath}"); // если захотите использовать \dfrac
pen frac_pen=fontsize(14pt); // более крупный шрифт для дробей 
dotfactor=8;
real rho(real t){return 0.074t+1.2;} // функция для полярной системы координат
xaxis("$x$", -1.7, 2.1, EndArrow(HookHead, 1.5mm)); // ось абсцисс
yaxis("$y$", -1.7, 1.7, EndArrow(HookHead, 1.5mm));  // ось ординат

real a, b, x1, x2, buff_left, buff_right;
a=-pi; // левая граница отрезка
b=2pi; // правая граница отрезка
x1=pi/6; // первый корень
x2=5pi/6; // второй корень
buff_left=pi/3; // продолжение спирали за точку a;
buff_right=pi/3; // продолжение спирали за точку b;

 // черная спираль
guide g=polargraph(rho, a-buff_left, b+buff_right, n=100, join=operator..);
draw(g, black); 
 
// синяя спираль
guide g2=polargraph(rho, a, b, n=100, join=operator..); 
draw(g2, blue+1bp); 

// Переходим от действительного числа к точке на плоскости
pair A, B, X1, X2;
A=polar(rho(a), a);
B=polar(rho(b), b); 
X1=polar(rho(x1), x1);  
X2=polar(rho(x2), x2);  
 
// Лейблы
label("$-\pi$", A, 0.5N, UnFill); 
label("$2\pi$", B, 0.5N, UnFill);
label("$\frac{\pi}{6}$", X1, 0.5NE, frac_pen);
label("$\frac{5\pi}{6}$", X2, 0.5NW, frac_pen);
labelx(0, 0.5SW);
dot(X1^^X2); // отметка точек
\end{asy}

\subsubsection*{Числовая прямая}
\vspace{1mm}
\begin{asy}
import graph;  //для числовой прямой и лейблов
import patterns;  // для штриховки
size(8cm);  // размер рисунка
dotfactor=8; // размер точек
real a=-4, b=3; // границы числовой прямой
real x1=-2, x2=1; // нули функции 
bool inside = false; // выделить промежуток между корнями? true/false
real buff=0.4; // зазор между стрелкой и окончанием штриховки
xaxis("$x$", a, b+buff, EndArrow(HookHead, 1.5mm));  // числовая прямая
labelx(x1); labelx(x2); // лейблы нулей
add("hatch", hatch (1mm, black)); // настройка штриховки
real h=0.4;  // высота заштрихованных прямоугольников

// прямоугольники
path left, center, right; 
left=(x1,0)--(x1,h)--(a,h)--(a,0)--cycle;
center=(x1,0)--(x1,h)--(x2,h)--(x2,0)--cycle;
right=(x2,0)--(x2,h)--(b,h)--(b,0)--cycle;


if(inside){
    filldraw(center, pattern ("hatch")); // отрисовка центральной области
}
else{
fill(left^^right, pattern ("hatch")); // штриховка
draw(subpath(left, 0, 2)); // контур для левой области
draw(subpath(right, 0, 2)); // контур для правой области
}

dot(x1 * E, filltype=UnFill); // выколотоая точка
dot(x2 * E); // закрашенная точка
\end{asy}

\subsubsection*{Часы}
\begin{asy}
defaultpen(fontsize(10pt));
size(5cm);
draw(unitcircle);
path seg=(0.9,0)--(0.95,0);
for (int i=0; i<12; ++i){  
    draw(rotate(30*i)*seg, black+1bp);
    label(string(i+1), 0.8*dir(60-30*i));
}
path arca=arc((0,0), r=1, angle1=60, angle2=150);
draw(arca, black+1bp);
\end{asy}

\subsubsection*{Касательная и секущая}
\begin{asy}
import geometry;
size(6cm);
defaultpen(fontsize(10pt));
pair P, A, B, C, C1, B1;
path circ=unitcircle;
P=(-2,-1); B=(0,-1);
C=point(circ, 0.7);

A=intersectionpoints(circ, P--C)[1];
C1=C+0.5unit(C-A);
B1=B+unit(B-P);
draw(circ^^B1--P--C1);
label("$A$", A, W+0.5N);
label("$B$", B, S);
label("$C$", C, N);
label("$P$", P, SW);
\end{asy}

\subsubsection*{Две окружности} 

\begin{asy}
import geometry;
size(7cm);
dotfactor=8;
pair A, B, C, D;
A=(0,0); B=(1,1); C=(-1,0); D=(1,-1);
draw(B--C--D--cycle^^A--C^^A--D^^A--B);
path circ=circle(A,sqrt(2));
path circ_=circle(C,sqrt(5));
draw(circ^^circ_); 
dot(A^^C, filltype=UnFill);
markangle(D, C, B, radius=0.5cm);
markrightangle(D, A, B, size=0.3cm);
label("$A$", A, NW);
label("$B$", B, NE);
label("$C$", C, NW);
label("$D$", D, SE);
\end{asy}


\subsubsection*{Четыре попарно касающихся шара, вписанных в цилиндр}

Чтобы отрендерить картинку, вставьте код, размещенный между командами begin и end ниже, в~\href{http://asymptote.ualberta.ca}{онлайн-версию Asymptote}. \\[3mm]

\includegraphics[width=5cm]{img/touching_spheres.pdf}

\begin{comment}
import solids;
size(5cm, 7cm);
settings.render=10;
currentprojection=orthographic(4,2,1);
triple A, B, C, D;
real h=sqrt(3), R=2h+3;
A=(-3,-h,0);
B=(3,-h,0);
C=(0,2h,0);
D=(0, 0, sqrt(36+24*h));

real op=0.7;
triple Q[]={A, B, C};
for (triple p : Q){
  draw(shift(p)*scale3(3)*unitsphere, blue+opacity(op));  
}
draw(shift(D)*scale3(R)*unitsphere, magenta+opacity(op));
revolution c=cylinder(-3Z,R,D.z+3+R);
draw(surface(c), magenta+opacity(0.1));

triple V[]={O, D, D+R*Z, -3Z};
path3 arca=arc(O,R*X,R*X,Z);
for (triple vec : V){
  draw(shift(vec)*arca, grey+0.3bp+opacity(1));
}
\end{comment}

\subsubsection*{Конус}
\begin{asy}
size(5cm);
import solids;
settings.render=0;
currentprojection=orthographic(0,6,2);
real h=2, r=1;
draw(cone(O, r, h, n=1));

path3 arca=shift(h*Z/2)*arc(O, 0.5X, -0.5X, Z);
path3 arca2=shift(h*Z/2)*arc(O, -0.5X, 0.5X, Z);
draw(arca); 
draw(arca2, dashed);
\end{asy}


\end{document}